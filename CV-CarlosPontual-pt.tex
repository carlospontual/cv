%%%%%%%%%%%%%%%%%%%%%%%%%%%%%%%%%%%%%%%%%
% Medium Length Graduate Curriculum Vitae
% LaTeX Template
% Version 1.1 (9/12/12)
%
% This template has been downloaded from:
% http://www.LaTeXTemplates.com
%
% Original author:
% Rensselaer Polytechnic Institute (http://www.rpi.edu/dept/arc/training/latex/resumes/)
%
% Important note:
% This template requires the res.cls file to be in the same directory as the
% .tex file. The res.cls file provides the resume style used for structuring the
% document.
%
%%%%%%%%%%%%%%%%%%%%%%%%%%%%%%%%%%%%%%%%%

%----------------------------------------------------------------------------------------
%	PACKAGES AND OTHER DOCUMENT CONFIGURATIONS
%----------------------------------------------------------------------------------------

\documentclass[margin, 10pt]{res} % Use the res.cls style, the font size can be changed to 11pt or 12pt here

\usepackage{helvet} % Default font is the helvetica postscript font
%\usepackage{newcent} % To change the default font to the new century schoolbook postscript font uncomment this line and comment the one above

\usepackage{hyperref}
\usepackage[utf8]{inputenc}
\usepackage[brazilian]{babel}
\usepackage{enumitem}
\setlist[itemize]{leftmargin=*}

\setlength{\textwidth}{5.1in} % Text width of the document

%
\def\name{Carlos Eduardo Pontual de Lemos Castro}
\def\endereco{Rua Silva Jatahy, 500 -- Ap 500 -- Meireles}
\def\cidade{Fortaleza, CE -- CEP: 60.165-070}
\def\mail{carloscode@gmail.com}
\def\skype{carlosedpontual }
\def\telefone{+55 85 9988-5104 / +55 85 3224-1180}
% -----

\renewcommand{\labelitemi}{$-$}


\hypersetup{
  colorlinks = true,
  urlcolor = black,
  pdfauthor = {\name},
  pdfkeywords = {resume, curriculum vitae},
  pdftitle = {\name: Curriculum Vitae-Carlos Eduardo Pontual},
  pdfsubject = {Curriculum Vitae-Carlos Eduardo Pontual},
  pdfpagemode = UseNone
}

\begin{document}

%----------------------------------------------------------------------------------------
%	NAME AND ADDRESS SECTION
%----------------------------------------------------------------------------------------

%\moveleft.5\hoffset\centerline{\large\bf John Smith} % Your name at the top
 
\moveleft1.0\hoffset\hbox{{\huge\bf Carlos Eduardo Pontual} {\huge de Lemos Castro }}

\vspace{0.10in}

\moveleft1.0\hoffset\hbox{
\begin{minipage}{0.55\linewidth}
  \endereco \\
  \cidade
\end{minipage}
\begin{minipage}{0.4\linewidth}
  \begin{tabular}{ll}
    \href{mailto:\mail}{\tt \mail} / \skype (Skype)\\
    Telefones: \telefone
  \end{tabular}
\end{minipage}
}

%\moveleft.5\hoffset\centerline{123 Broadway} % Your address
%\moveleft.5\hoffset\centerline{City, State 12345}
%\moveleft.5\hoffset\centerline{(000) 111-1111 or (111) 111-1112}

%----------------------------------------------------------------------------------------

\begin{resume}

%----------------------------------------------------------------------------------------
%	OBJECTIVE SECTION
%----------------------------------------------------------------------------------------
 \vspace{-0.12in}
\section{Objetivo}  

Engenheiro de Software

%----------------------------------------------------------------------------------------
%	RESUME SECTION
%----------------------------------------------------------------------------------------
 
\section{Resumo}  

Líder técnico. Bacharel e mestre em Ciência da Computação, com ênfase em Engenharia de Software. Entusiasta de novas tecnologias e apaixonado por desafios. Tem larga experiência em desenvolvimento de aplicações para dispositivos móveis, tendo trabalhado com diversas plataformas. Atualmente desenvolve jogos e aplicativos para dispositivos Android.

%----------------------------------------------------------------------------------------
%	EDUCATION SECTION
%----------------------------------------------------------------------------------------

\section{Formação \\ Acadêmica}

{\sl\bf Mestrado em Ciência da Computação} \hfill {\bf Recife, jan 2008 -- jun 2010} \\
Centro de Informática (CIn) / Universidade Federal de Pernambuco (UFPE) %\\
%Título da dissertação: {\sl Design Rules for Increasing Modularity with CaesarJ}
\vspace{0.02in}
\\{\sl\bf Bacharelado em Ciência da Computação} \hfill {\bf Fortaleza, jan 2004 -- dez 2007} \\
Universidade Federal do Ceará (UFC)  
 
 

%----------------------------------------------------------------------------------------
%	PROFESSIONAL EXPERIENCE SECTION
%----------------------------------------------------------------------------------------
 
\section{Experiência \\ Profissional}

{\sl\bf Grupo de Redes, Engenharia de Software e Sistemas (GREat)}  \\
{\sl Analista de Sistemas} \hfill {\bf Fortaleza, dez 2010 -- atual}
\begin{itemize} \itemsep -2pt % Reduce space between items
\item Líder técnico em dois projetos de desenvolvimento de aplicativos para Android
\item Desenvolvimento de aplicativos e jogos para dispositivos Android
\item Desenvolvimento de testes automáticos para Android
\item Análise e desenvolvimento de adaptações de sistemas embarcados para operadoras de celulares – Feature Phones (C/C++) / Smart Phones (Android)
\end{itemize}
 
{\sl\bf doistempos.com} \\
{\sl Consultor} \hfill {\bf Fortaleza, fev 2007 -- jan 2008}
\begin{itemize} \itemsep -2pt 
\item Desenvolvimento da aplicação doistempos.com Mobile (JME)
\item Desenvolvimento de aplicativos móveis utilizando o .NET Compact Framework (C\#, Web Services, SQL Server)
\end{itemize} 


{\sl\bf Grupo de Redes, Engenharia de Software e Sistemas (GREat)}  \\
{\sl Bolsista de Iniciação Científica (PIBIC/CNPq)} \hfill {\bf Fortaleza, jan 2005 -- jul 2007}
\begin{itemize} \itemsep -2pt % Reduce space between items
\item Desenvolvimento da aplicação para dispositivos móveis UbiPEP – Prontuário eletrônico ubíquo (Java – JME)
\item Manutenção do Sistema de Controle e Gerência de Redes (PHP, MySQL)
\item Desenvolvimento da Ferramenta GET-RNA – Gerador de Entradas Para Treinamento de Redes Neurais (Java - JSE)
\end{itemize}

%----------------------------------------------------------------------------------------
%	OTHER PROJECTS
%---------------------------------------------------------------------------------------- 

\section{Outras \\ Atividades}

{\sl\bf Xlung} \\
{\sl Analista de Sistemas e Consultor} \hfill {\bf Fortaleza, mai 2012 -- atual}
\begin{itemize} \itemsep -2pt 
\item Análise, projeto e desenvolvimento do portal xlung.net (Ruby On Rails / Passenger)
\end{itemize} 

\section{Cursos}
Asp.NET MVC 2.0, 40h \\ 
OpenGL ES, 16h \\ 
Windows Phone, 32h

%----------------------------------------------------------------------------------------
%	COMPUTER SKILLS SECTION
%----------------------------------------------------------------------------------------

\section{Conhecimentos}
{\sl\bf Linguagens de Programação:}  Android, Java, Ruby, C\#, C, C++, Python, PHP \\
{\sl\bf Frameworks:} Ruby on Rails, Qt, .NET MVC 2.0, .NET Compact Framework \\
{\sl\bf Bancos de dados:} MySQL, SQLite e SQL Server \\
Desenvolvimento nos {\sl\bf Sistemas operacionais} Linux, Windows e Mac (OSX) \\
{\sl\bf Sistemas de controle de versão} SVN, ClearCase e Git \\
{\sl\bf Metodologias de desenvolvimento ágeis} (Scrum e XP)  \\

\section {Informações Adicionais}
Brasileiro, 27 anos, solteiro, sem filhos \\
{\sl\bf Idiomas:} Português (Nativo), Inglês (Fluente) \\
{\sl\bf Certificação} {\sl Oracle Certified Java Programmer 6} \\
{\sl\bf Curriculo Lattes:} \url{http://lattes.cnpq.br/9333882757346321} \\
{\sl\bf LinkedIn:} \url{http://linkedin.com/pub/carlos-eduardo-pontual/4/17a/291} \\

%\begin{itemize} \itemsep -2pt
%\item Android
%\item Ruby on Rails
%\item Java para Desktops (JSE) e dispositivos móveis (JME)
%\item C, C++ (Qt), PHP, Python
%\item .NET (C\#, MVC 2.0) e .NET Compact Framework (C\#)
%\item Bancos de dados MySQL, SQLite e SQL Server
%\item Sistemas operacionais Linux, Windows e Mac (OSX)
%\item Sistemas de controle de versão SVN, ClearCase e Git
%\item Metodologias de desenvolvimento ágeis (Scrum e XP) 
%\end{itemize}


%----------------------------------------------------------------------------------------


\end{resume}
\end{document}